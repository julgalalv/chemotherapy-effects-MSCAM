\section{Appendix A}
%-----------------------------------------------------------------------------
% EQUATIONS
%-----------------------------------------------------------------------------

\section{Equations}
\label{sec:eqs}
This section gives some examples of writing mathematical equations in your thesis.

Maxwell's equations read:
\begin{subequations}
	\label{eq:maxwell}
	\begin{align}[left=\empheqlbrace]
		\nabla\cdot \bm{D} & = \rho, \label{eq:maxwell1} \\
		\nabla \times \bm{E} +  \frac{\partial \bm{B}}{\partial t} & = \bm{0}, \label{eq:maxwell2} \\
		\nabla\cdot \bm{B} & = 0, \label{eq:maxwell3} \\
		\nabla \times \bm{H} - \frac{\partial \bm{D}}{\partial t} &= \bm{J}. \label{eq:maxwell4}
	\end{align}
\end{subequations}

Equation~\eqref{eq:maxwell} is automatically labeled by \texttt{cleveref},
as well as Equation~\eqref{eq:maxwell1} and Equation~\eqref{eq:maxwell3}.
Thanks to the \verb|cleveref| package, there is no need to use \verb|\eqref|.
Equations have to be numbered only if they are referenced in the text.

Equations~\eqref{eq:maxwell_multilabels1}, \eqref{eq:maxwell_multilabels2}, \eqref{eq:maxwell_multilabels3}, and \eqref{eq:maxwell_multilabels4} show again Maxwell's equations without brace:
\begin{align}
	\nabla\cdot \bm{D} & = \rho, \label{eq:maxwell_multilabels1} \\
	\nabla \times \bm{E} +  \frac{\partial \bm{B}}{\partial t} &= \bm{0}, \label{eq:maxwell_multilabels2} \\
	\nabla\cdot \bm{B} & = 0, \label{eq:maxwell_multilabels3} \\
	\nabla \times \bm{H} - \frac{\partial \bm{D}}{\partial t} &= \bm{J} \label{eq:maxwell_multilabels4}.
\end{align}

Equation~\eqref{eq:maxwell_singlelabel} is the same as before,
but with just one label:
\begin{equation}
	\label{eq:maxwell_singlelabel}
	\left\{
	\begin{aligned}
		\nabla\cdot \bm{D} & = \rho, \\
		\nabla \times \bm{E} +  \frac{\partial \bm{B}}{\partial t} &= \bm{0},\\
		\nabla\cdot \bm{B} & = 0, \\
		\nabla \times \bm{H} - \frac{\partial \bm{D}}{\partial t} &= \bm{J}.
	\end{aligned}
	\right.
\end{equation}

%-----------------------------------------------------------------------------
% FIGURES, TABLES AND ALGORITHMS
%-----------------------------------------------------------------------------
\section{Figures, Tables and Algorithms}

Figures, Tables and Algorithms have to contain a Caption that describes their content, and have to be properly referred in the text.

\subsection{Figures}
\label{subsec:figures}

For including pictures in your text you can use \texttt{TikZ} for high-quality hand-made figures \cite{tikz},
or just include them with the command
\begin{verbatim}
	\includegraphics[options]{filename.xxx}
\end{verbatim}
Here xxx is the correct format, e.g.  \verb|.png|, \verb|.jpg|, \verb|.eps|, \dots.

\begin{figure}[H]
	\centering
	\includegraphics[width=0.3\textwidth]{logo_polimi_scritta.eps}
	\caption{Caption of the Figure.}
	\label{fig:quadtree}
\end{figure}

Thanks to the \texttt{\textbackslash subfloat} command, a single figure, such as Figure~\ref{fig:quadtree},
can contain multiple sub-figures with their own caption and label, e.g. Figure~\ref{fig:polimi_logo1} and Figure~\ref{fig:polimi_logo2}. 

\begin{figure}[H]
	\centering
	\subfloat[One PoliMi logo.\label{fig:polimi_logo1}]{
		\includegraphics[scale=0.5]{Images/logo_polimi_scritta.eps}
	}
	\quad
	\subfloat[Another one PoliMi logo.\label{fig:polimi_logo2}]{
		\includegraphics[scale=0.5]{Images/logo_polimi_scritta2.eps}
	}
	\caption[]{Caption of the Figure.}
	\label{fig:quadtree2}
\end{figure}

\subsection{Tables}
\label{subsec:tables}

Within the environments \texttt{table} and  \texttt{tabular} you can create very fancy tables as the one shown in Table~\ref{table:example}.

\begin{table}[H]
	\caption*{\textbf{Example of Table (optional)}}
	\centering 
	\begin{tabular}{|p{3em} c c c |}
		\hline
		\rowcolor{bluePoli!40}
		& \textbf{column1} & \textbf{column2} & \textbf{column3} \T\B \\
		\hline \hline
		\textbf{row1} & 1 & 2 & 3 \T\B \\
		\textbf{row2} & $\alpha$ & $\beta$ & $\gamma$ \T\B\\
		\textbf{row3} & alpha & beta & gamma \B\\
		\hline
	\end{tabular}
	\\[10pt]
	\caption{Caption of the Table.}
	\label{table:example}
\end{table}

You can also consider to highlight selected columns or rows in order to make tables more readable.
Moreover, with the use of \texttt{table*} and the option \texttt{bp} it is possible to align them at the bottom of the page. One example is presented in Table~\ref{table:exampleC}. 

\begin{table*}[bp]
	\centering 
	\begin{tabular}{|p{3em} | c | c | c | c | c | c|}
		\hline
		%    \rowcolor{bluePoli!40}
		& \textbf{column1} & \textbf{column2} & \textbf{column3} & \textbf{column4} & \textbf{column5} & \textbf{column6} \T\B \\
		\hline \hline
		\textbf{row1} & 1 & 2 & 3 & 4 & 5 & 6 \T\B\\
		\textbf{row2} & a & b & c & d & e & f \T\B\\
		\textbf{row3} & $\alpha$ & $\beta$ & $\gamma$ & $\delta$ & $\phi$ & $\omega$ \T\B\\
		\textbf{row4} & alpha & beta & gamma & delta & phi & omega \B\\
		\hline
	\end{tabular}
	\\[10pt]
	\caption{Highlighting the columns}
	\label{table:exampleC}
\end{table*}

\subsection{Algorithms}
\label{subsec:algorithms}

Pseudo-algorithms can be written in \LaTeX{} with the \texttt{algorithm} and \texttt{algorithmic} packages.
An example is shown in Algorithm~\ref{alg:var}.
\begin{algorithm}[H]
	\label{alg:example}
	\caption{Name of the Algorithm}
	\label{alg:var}
	\label{protocol1}
	\begin{algorithmic}[1]
		\STATE Initial instructions
		\FOR{$for-condition$}
		\STATE{Some instructions}
		\IF{$if-condition$}
		\STATE{Some other instructions}
		\ENDIF
		\ENDFOR
		\WHILE{$while-condition$}
		\STATE{Some further instructions}
		\ENDWHILE
		\STATE Final instructions
	\end{algorithmic}
\end{algorithm} 

\section{Some further useful suggestions}

Theorems have to be formatted as follows:
\begin{theorem}
	\label{a_theorem}
	Write here your theorem. 
\end{theorem}
\textit{Proof.} If useful you can report here the proof.
\vspace{0.3cm} % Insert vertical space

Propositions have to be formatted as follows:
\begin{proposition}
	Write here your proposition.
\end{proposition}
\vspace{0.3cm} % Insert vertical space

How to insert itemized lists:
\begin{itemize}
	\item first item;
	\item second item.
\end{itemize}
How to write numbered lists:
\begin{enumerate}
	\item first item;
	\item second item.
\end{enumerate}

\section{Use of copyrighted material}

Each student is responsible for obtaining copyright permissions, if necessary, to include published material in the thesis.
This applies typically to third-party material published by someone else.

\section{Plagiarism}

You have to be sure to respect the rules on Copyright and avoid an involuntary plagiarism.
It is allowed to take other persons' ideas only if the author and his original work are clearly mentioned.
As stated in the Code of Ethics and Conduct, Politecnico di Milano \textit{promotes the integrity of research,
	condemns manipulation and the infringement of intellectual property}, and gives opportunity to all those
who carry out research activities to have an adequate training on ethical conduct and integrity while doing research.
To be sure to respect the copyright rules, read the guides on Copyright legislation and citation styles available
at:
\begin{verbatim}
	https://www.biblio.polimi.it/en/tools/courses-and-tutorials
\end{verbatim}
You can also attend the courses which are periodically organized on "Bibliographic citations and bibliography management".

